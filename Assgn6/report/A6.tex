\documentclass{article}

\usepackage{graphicx}
\usepackage{subfigure}
\usepackage[hypcap]{caption}
\usepackage{listings}
\usepackage{float}
\floatstyle{plaintop}
\restylefloat{table}

\title{Experimental Design and Data Analysis: Assignment 6}
\author{Andrew Bedard(2566978) \& Simone van Gompel(2567525) \\ Group 19}

\begin{document}

  \maketitle

  \section*{Exercise 1}
    \subsection*{1}
    \subsection*{2}
    \subsection*{3}
    \subsection*{4}
    \subsection*{5}
    \subsection*{6}
    \subsection*{7}
    \subsection*{8}
    \subsection*{9}
    \subsection*{10}
    
  \section*{Exercise 2}
    \subsection*{1}
    The data contained in \textit{psi.txt} was read in, the following figures were obtained.
    \subsection*{2}
    Fitting a logistic regression model with psi and gpa as explanatory variables for the outcome being that the student passed their assessment or not, we obtain the following table:
    	\begin{figure}[H]
    	\begin{lstlisting}[language=R]
	Coefficients:
            Estimate Std. Error z value Pr(>|z|)   
(Intercept)  -11.602      4.213  -2.754  0.00589 **
psi            2.338      1.041   2.246  0.02470 * 
gpa            3.063      1.223   2.505  0.01224 * 
---
Signif. codes:  
0 ‘***’ 0.001 ‘**’ 0.01 ‘*’ 0.05 ‘.’ 0.1 ‘ ’ 1
    	\end{lstlisting}
    	\caption{Parameter estimation for logistic regression model}
    	\label{fig:log_reg}
    \end{figure}
    
    Thus we determine that our logistic regression model should be:
    \begin{equation}
    \frac{Pr(\textit{pass = 1})}{Pr(\textit{pass = 0})} = \exp(-11.602 + 2.338*\textit{psi} + 3.063*\textit{gpa})
    \end{equation}
    \subsection*{3}
    Based on the p-value obtained in Figure:\ref{fig:log_reg}, we reject the null hypothesis that there is no effect of psi on the outcomes of the students final assessment. Further based on our parameters for the logistic regression model, we see that a positive value, ie. 1, for \textit{psi} causes an increase in probability of passing, so we conclude that \textit{psi} does in fact work.
    \subsection*{4}
    To estimate the probability that a student with a \textit{gpa} equal to 3 who receives \textit{psi} passes the assignment, we simply enter our values into equation 1, our logistic regression model.
    \[
    \frac{Pr(\textit{pass = 1})}{Pr(\textit{pass = 0})} = \exp(-11.602 + 2.338*(1) + 3.063*(3)) = 0.9277
    \]
    So there is a 92.77 \% chance of a student with \textit{gpa} of 3 who receives \textit{psi} of passing the final assignment.
    \subsection*{5}
    \subsection*{6}
    \subsection*{7}
    \subsection*{8}

  \section*{Exercise 3}
    \subsection*{1}
    \subsection*{2}
    \subsection*{3}
    \subsection*{4}
    \subsection*{5}
    
  \section{R-Code}
    \subsection{Exercise 1}\label{sec:RE1}
      \begin{lstlisting}[language=R]
      \end{lstlisting}
    \subsection{Exercise 2}\label{sec:RE2}
      \begin{lstlisting}[language=R]
      \end{lstlisting}
    \subsection{Exercise 3}\label{sec:RE3}
      \begin{lstlisting}[language=R]
      \end{lstlisting}
    \subsection{Exercise 4}\label{sec:RE4}
      \begin{lstlisting}[language=R]
      \end{lstlisting}
\end{document}
